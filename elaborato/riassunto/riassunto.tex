% Options for packages loaded elsewhere
\PassOptionsToPackage{unicode}{hyperref}
\PassOptionsToPackage{hyphens}{url}
\documentclass[
]{article}
\usepackage{xcolor}
\usepackage[margin=1in]{geometry}
\usepackage{amsmath,amssymb}
\setcounter{secnumdepth}{-\maxdimen} % remove section numbering
\usepackage{iftex}
\ifPDFTeX
  \usepackage[T1]{fontenc}
  \usepackage[utf8]{inputenc}
  \usepackage{textcomp} % provide euro and other symbols
\else % if luatex or xetex
  \usepackage{unicode-math} % this also loads fontspec
  \defaultfontfeatures{Scale=MatchLowercase}
  \defaultfontfeatures[\rmfamily]{Ligatures=TeX,Scale=1}
\fi
\usepackage{lmodern}
\ifPDFTeX\else
  % xetex/luatex font selection
\fi
% Use upquote if available, for straight quotes in verbatim environments
\IfFileExists{upquote.sty}{\usepackage{upquote}}{}
\IfFileExists{microtype.sty}{% use microtype if available
  \usepackage[]{microtype}
  \UseMicrotypeSet[protrusion]{basicmath} % disable protrusion for tt fonts
}{}
\makeatletter
\@ifundefined{KOMAClassName}{% if non-KOMA class
  \IfFileExists{parskip.sty}{%
    \usepackage{parskip}
  }{% else
    \setlength{\parindent}{0pt}
    \setlength{\parskip}{6pt plus 2pt minus 1pt}}
}{% if KOMA class
  \KOMAoptions{parskip=half}}
\makeatother
\usepackage{graphicx}
\makeatletter
\newsavebox\pandoc@box
\newcommand*\pandocbounded[1]{% scales image to fit in text height/width
  \sbox\pandoc@box{#1}%
  \Gscale@div\@tempa{\textheight}{\dimexpr\ht\pandoc@box+\dp\pandoc@box\relax}%
  \Gscale@div\@tempb{\linewidth}{\wd\pandoc@box}%
  \ifdim\@tempb\p@<\@tempa\p@\let\@tempa\@tempb\fi% select the smaller of both
  \ifdim\@tempa\p@<\p@\scalebox{\@tempa}{\usebox\pandoc@box}%
  \else\usebox{\pandoc@box}%
  \fi%
}
% Set default figure placement to htbp
\def\fps@figure{htbp}
\makeatother
% definitions for citeproc citations
\NewDocumentCommand\citeproctext{}{}
\NewDocumentCommand\citeproc{mm}{%
  \begingroup\def\citeproctext{#2}\cite{#1}\endgroup}
\makeatletter
 % allow citations to break across lines
 \let\@cite@ofmt\@firstofone
 % avoid brackets around text for \cite:
 \def\@biblabel#1{}
 \def\@cite#1#2{{#1\if@tempswa , #2\fi}}
\makeatother
\newlength{\cslhangindent}
\setlength{\cslhangindent}{1.5em}
\newlength{\csllabelwidth}
\setlength{\csllabelwidth}{3em}
\newenvironment{CSLReferences}[2] % #1 hanging-indent, #2 entry-spacing
 {\begin{list}{}{%
  \setlength{\itemindent}{0pt}
  \setlength{\leftmargin}{0pt}
  \setlength{\parsep}{0pt}
  % turn on hanging indent if param 1 is 1
  \ifodd #1
   \setlength{\leftmargin}{\cslhangindent}
   \setlength{\itemindent}{-1\cslhangindent}
  \fi
  % set entry spacing
  \setlength{\itemsep}{#2\baselineskip}}}
 {\end{list}}
\usepackage{calc}
\newcommand{\CSLBlock}[1]{\hfill\break\parbox[t]{\linewidth}{\strut\ignorespaces#1\strut}}
\newcommand{\CSLLeftMargin}[1]{\parbox[t]{\csllabelwidth}{\strut#1\strut}}
\newcommand{\CSLRightInline}[1]{\parbox[t]{\linewidth - \csllabelwidth}{\strut#1\strut}}
\newcommand{\CSLIndent}[1]{\hspace{\cslhangindent}#1}
\setlength{\emergencystretch}{3em} % prevent overfull lines
\providecommand{\tightlist}{%
  \setlength{\itemsep}{0pt}\setlength{\parskip}{0pt}}
\usepackage[a-1b]{pdfx}
\usepackage{hyperref}    % Required for metadata
\hypersetup{
  pdftitle={Elaborazione di un dataset degli estremi di temperatura mensili per il territorio italiano},
  pdfauthor={Davide Nicoli},
  pdfsubject={Riassunto della tesi di laurea},
  unicode=true,
  colorlinks=false,
  pdfborder={0 0 0},
  pdfpagemode=UseNone,
}
\usepackage{bookmark}
\IfFileExists{xurl.sty}{\usepackage{xurl}}{} % add URL line breaks if available
\urlstyle{same}
\hypersetup{
  pdfauthor={Davide Nicoli},
  hidelinks,
  pdfcreator={LaTeX via pandoc}}

\author{Davide Nicoli}
\date{2025-03-24}

\begin{document}

\subsection{Introduzione}\label{introduzione}

Il lavoro ha avuto l'obiettivo di elaborare un dataset degli estremi
minimi e massimi giornalieri di temperatura relativi al trentennio
1991-2020 per le serie registrate nel centro-nord italia, nell'ambito di
un più ampio progetto volto alla costruzione di un dataset ad alta
risoluzione delle normali climatologiche esteso all'intero territorio
italiano. In particolare si mira ad aggiornare il lavoro svolto in
Brunetti et al. (2014).

Il dataset risultante è stato costruito attraverso la raccolta, il
controllo e l'integrazione di dati ed anagrafiche di stazione
provenienti da diverse fonti provinciali, regionali e nazionali.

Dopo aver finalizzato la raccolta si è proceduto ad una valutazione di
massima della capacità del modello elaborato nel lavoro precedente di
interpolare le climatologie del nuovo dataset.

\subsection{Metodologia}\label{metodologia}

\begin{enumerate}
\def\labelenumi{\arabic{enumi}.}
\tightlist
\item
  Raccolta dei dati: sono stati acquisiti dati da vari enti, tra cui
  ARPA, ISPRA, ISAC-CNR e Dipartimento di Protezione Civile, prestando
  particolare attenzione a coprire il periodo 1991-2020 e recuperando i
  dati rimanenti dove farlo non fosse troppo dispendioso in termini di
  tempo. La frammentazione e l'eterogeneità delle fonti hanno richiesto
  un attento lavoro di standardizzazione di dati e anagrafiche;
\item
  Merging: le serie di dati sono state integrate attraverso un processo
  che ha combinato fonti diverse per garantire una maggiore copertura
  spaziale e temporale ed eliminare le serie duplicate;
\item
  Controllo qualità: è stata eseguita un'analisi dei dati e metadati
  delle serie per individuare errori e incongruenze. In particolare, per
  ogni serie di dati si è fatto un confronto con la rianalisi ERA5 e con
  le serie meteorologiche limitrofe tramite tecniche studiate in altri
  lavori (Mitchell and Jones 2005; Di Luzio et al. 2008); per i metadati
  invece si è fatto un controllo manuale nei casi sospetti;
  4.Valutazione del modello di interpolazione: ogni serie del dataset
  soddisfacente determinati criteri di qualità minimi e completata dei
  dati mancanti è stata confrontata con la corrispondente serie
  sintetica ricostruita tramite interpolazione geospaziale delle serie
  limitrofe con Local Weighted Linear Regression.
\end{enumerate}

\subsection{Risultati}\label{risultati}

\begin{itemize}
\tightlist
\item
  Il dataset finale comprende 2501 serie sul territorio del centro-nord
  Italia, più 1294 nelle zone limitrofe dell'arco alpino;
\item
  è stata riscontrata una discrepanza maggiore tra modello ed
  osservazioni nelle temperature minime rispetto alle massime,
  particolarmente evidente in aree montane e costiere, in determinate
  regioni più che in altre;
\item
  l'analisi ha evidenziato la necessità di studiare in maniera più
  approfondita la procedura di interpolazione spaziale dei dati
  aggiungendo al modello ulteriori algoritmi per meglio catturare il
  legame tra le caratteristiche geografiche del territorio e le normali
  climatiche.
\end{itemize}

\subsection{Conclusioni}\label{conclusioni}

L'integrazione delle diverse fonti ha permesso di ottenere un dataset
con disponibilità di dati molto maggiore di quelli preesistenti per lo
studio del clima italiano. Tuttavia, permangono alcune criticità, come
la difficoltà di omogeneizzare dati provenienti da strumenti e
metodologie differenti.

\protect\phantomsection\label{refs}
\begin{CSLReferences}{1}{0}
\bibitem[\citeproctext]{ref-brunettiHighresolutionTemperatureClimatology2014}
Brunetti, Michele, Maurizio Maugeri, Teresa Nanni, C. Simolo, and J.
Spinoni. 2014. {``High-Resolution Temperature Climatology for {Italy}:
Interpolation Method Intercomparison.''} \emph{International Journal of
Climatology} 34 (4): 1278--96. \url{https://doi.org/10.1002/joc.3764}.

\bibitem[\citeproctext]{ref-diluzioConstructingRetrospectiveGridded2008}
Di Luzio, Mauro, Gregory L. Johnson, Christopher Daly, Jon K. Eischeid,
and Jeffrey G. Arnold. 2008. {``Constructing {Retrospective Gridded
Daily Precipitation} and {Temperature Datasets} for the {Conterminous
United States} in: {Journal} of {Applied Meteorology} and {Climatology
Volume} 47 {Issue} 2 (2008).''} \emph{Journal of Applied Meteorology and
Climatology} 47 (2): 475--97.
\url{https://doi.org/10.1175/2007JAMC1356.1}.

\bibitem[\citeproctext]{ref-mitchellImprovedMethodConstructing2005}
Mitchell, Timothy D., and Philip D. Jones. 2005. {``An Improved Method
of Constructing a Database of Monthly Climate Observations and
Associated High-Resolution Grids.''} \emph{International Journal of
Climatology} 25 (6): 693--712. \url{https://doi.org/10.1002/joc.1181}.

\end{CSLReferences}

\end{document}
