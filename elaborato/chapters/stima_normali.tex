% !TEX root = ../main.tex

Scopo di questa parte del lavoro è, partendo da serie di dati di stazione grezzi o già aggregati a livello giornaliero o mensile, produrre le normali climatologiche mensili georiferite sulle stazioni di rilevazione, ovvero medie delle temperature massime e minime giornaliere per ogni mese dell'anno collocate nello spazio. Il dataset risultante pertanto consterà di due tabelle: quella dei metadati di ogni serie (almeno latitudine, longitudine, quota, identificativo della serie) e quella dei contenuti delle serie (mese a cui la stima si riferisce, identificativo della serie di appartenenza, valori di temperatura minima e massima in \(\unit{\degreeCelsius}\)).

Il procedimento, come da linee guida~\cite{WMOGuidelinesNormalsCalculation2017}, consta di quattro passaggi:

\begin{enumerate}
  \item raccolta dei dati: trasformazione dei dati come forniti dai gestori
    delle reti di misura in serie giornaliere di estremi di temperatura;
  \item controllo qualità: eliminazione degli errori presenti nelle serie
    giornaliere tramite analisi puntuale dei dati e confronto con dati
    limitrofi;
  \item omogeneizzazione: individuazione e correzione delle deviazioni che il
    cambiamento di strumentazione o ambiente circostante introducono nelle
    serie;
  \item aggregazione su scala mensile.
\end{enumerate}

% TODO: controllare questo paragrafo
Differentemente dalle linee guida WMO, si è scelto di tenere come buone le serie di dati che registrano almeno cinque anni con mesi validi (almeno 20 giorni di dati validi per ogni mese, non più di 4 giorni consecutivi mancanti).

\section{Raccolta dei dati}\label{ch:raccolta-dati}
% !TEX root = ../../main.tex

Nella sezione seguente si presenta da dove sono stati reperiti i dati di partenza e quali scelte sono state fatte nella costruzione di serie di estremi di temperatura giornalieri.

\subsection{Disponibilità e qualità di dati e metadati}
Al fine di preparare il dataset delle climatologie si è reso necessario raccogliere, controllare e combinare i dati rilevati dalle stazioni meteorologiche distribuite sul territorio italiano e nelle zone limitrofe all'arco alpino, in modo da avere a disposizione serie giornaliere di estremi di temperatura. Tale operazione è risultata particolarmente complessa, per le misure effettuate nel trentennio 1991--2020, a causa della frammentarietà e disomogeneità delle sorgenti di dati, attribuibili a tre ragioni principali. Innanzitutto, vi è un elevato numero di soggetti coinvolti nella gestione delle principali reti di rilevazione meteorologica e dei dati da esse derivanti. Questo fenomeno ha avuto origine nel trasferimento di competenze del Sistema Idrografico e Mareografico Nazionale (SIMN) a centri funzionali, Dipartimento di Protezione Civile (DPC) e Agenzie Regionali per la Protezione dell'Ambiente (ARPA)~\cite{InquadramentoStoricoMonitoraggio} avvenuto all'inizio degli anni 2000. Tali enti ed agenzie non seguono un protocollo nazionale che omogenizzi i criteri di gestione della rete e dell'elaborazione e distribuzione dei dati, per cui persino raccogliere i dataset risulta un problema non banale. In secondo luogo, dalla fine degli anni '80 vi è stata una progressiva diffusione dei sistemi di rilevamento automatico, accompagnata dalla dismissione di quelli meccanici. Questo cambiamento ha introdotto significative disomogeneità nelle serie temporali e spesso ha portato al loro troncamento, laddove la stazione di riferimento non sia stata rimpiazzata. Infine si riscontra una generale carenza di documentazione relativa alle stazioni stesse: di norma si trovano liste di anagrafiche, posizioni e quote (non di rado imprecise), ma solo pochi enti forniscono uno storico delle modifiche apportate alle stazioni (eventuali ricollocamenti, cambiamenti nelle tipologie di sensori, ecc.).

La maggior parte dei dati raccolti sono stati forniti come aggregazioni a livello giornaliero, solo in poche situazioni si è fatto ricorso ai dati grezzi. Da un'analisi preliminare è risultato evidente che ogni rete o gestore di dataset utilizza un proprio criterio nel calcolare tali stime a partire dai dati rilevati dai sensori. In particolare, si sono incontrate scelte differenti riguardo i seguenti ambiti:

\begin{itemize}
  \item
    definizione di ``giornata di misura'': in alcuni casi va dalle \DTMdisplaytime{09}{00}{} del giorno corrente alle \DTMdisplaytime{09}{00}{} del giorno seguente (approccio tradizionale), in altri va dalle \DTMdisplaytime{00}{00}{} alle \DTMdisplaytime{00}{00}{} (approccio moderno);
  \item
    timezone delle date: vengono utilizzati GMT o CET;
  \item
    aggregazione dei dati grezzi: vengono forniti gli estremi assoluti delle temperature rilevate nel corso della giornata (in linea con quanto cercato per la tesi) o gli estremi delle medie orarie (sempre distanti al più qualche decimo di grado dagli estremi assoluti e sempre ``meno estremi'' rispetto ad essi).
\end{itemize}

La disponibilità temporale e spaziale delle serie nei singoli dataset infine risulta essere perlopiù incompleta: ad esempio dataset nazionali come SCIA, quello del DPC e del CNR-ISAC risultano essere incompleti sulla dimensione della distribuzione spaziale delle serie, mentre quelli regionali non registrano le misure antecedenti al passaggio di competenze citato in precedenza.

La presenza delle problematiche illustrate e la necessità di serie giornaliere il più possibile complete e consistenti sotto il profilo sia temporale che spaziazle hanno reso necessaria l'elaborazione di procedure adeguate di meging dei singoli dataset e saranno presentate nella sezione~\ref{ch:merging}.

\subsection{Sorgenti utilizzate}\label{ch:sources}
Di seguito i dataset impiegati per il lavoro. Nell'appendice~\ref{app:datasets} viene fornita una descrizione dettagliata di contenuti, problematiche e disponibilità dati di ciascuno di essi.

\begin{table}[h]
  \centering
  \begin{tabular}{l l}
    \toprule
    Nome & Copertura \\
    \midrule
    SCIA & Italia \\
    DPC & Italia \\
    CNR-ISAC & Italia \\
    ARPA Piemonte & Regione \\
    ARPA Lombardia & Regione \\
    ARPA Veneto & Regione \\
    MeteoTrentino & Provincia (Trento) \\
    CIVIS Bolzano & Provincia \\
    ARPA FVG/OSMER & Regione \\
    ARPAL & Regione (Liguria) \\
    SIR Toscana & Regione \\
    Dext3r & Regione (Emilia-Romagna) \\
    ARPAM & Regione (Marche) \\
    ARPA Umbria & Regione \\
    \bottomrule
  \end{tabular}
\end{table}

A livello qualitativo si è constatato che:

\begin{itemize}
  \item
    i dataset regionali sono i meglio forniti in quanto a disponibilità dei dati recenti (post-2000) e qualità dei metadati delle stazioni;
  \item
    SCIA è ben fornito di dati storici (pre-2000), mentre quelli recenti sono incompleti e generalmente collocati in maniera poco accurata; pur essendo passate da un'operazione di quality-check alcune serie contengono significativi errori;
  \item
    CNR-ISAC fornisce qualche centinaio di serie lunghe già omogeneizzate;
  \item
    DPC registra serie spesso non presenti negli altri dataset, tuttavia fornisce gli estremi giornalieri delle medie orarie invece che gli estremi assoluti e le stazioni sono spesso collocate in maniera poco accurata.
\end{itemize}

\subsection{Merging}\label{ch:merging}
La procedura di merging ha per scopo la combinazione delle varie sequenze di dati in serie rappresentative della collocazione geografica delle stazioni di rilevamento. Nell'elaborarla si è prestata particolare attenzione a due questioni: l'unione delle sequenze duplicate (merging ``orizzontale'') e l'aggregazione di sequenze rappresentative di una stessa località geografica ma provenienti da stazioni o sensori diversi (merging ``verticale''). Né il primo né il secondo problema hanno avuto soluzione immediata a causa della mancanza di codici di identificazione univoci, delle inaccuratezze sul collocamento geografico delle stazioni e di varie incongruenze nella registrazione dei valori di temperatura.

Il procedimento definitivo si articola in due fasi: l'identificazione delle sequenze facenti parte della stessa serie e l'unione di queste in singole serie.

\subsubsection{Identificazione delle serie}
Se si considerano le sequenze recuperate dai vari dataset come i vertici di un grafo le cui connessioni sono costituite dalla relazione ``fanno parte della stessa serie'' (da qui in poi ``\emph{match}''), il problema in esame ha come soluzione le componenti connesse del suddetto grafo. Questo approccio consente di definire la relazione di connessione, che in generale potrebbe essere molto articolata, in maniera più semplice, e gestire in maniera approppriata il raggruppamento di più sequenze anche quando non ci sono \emph{match} tra tutte le componenti e ogni altra.

Idealmente, per individuare un \emph{match}, basterebbe controllare l'uguaglianza delle sequenze di dati attribuite ai sensori nei vari dataset, o degli identificativi univoci delle stazioni, oppure la coincidenza o prossimità delle posizioni. Non è però questo il caso per i cataloghi a disposizione. In particolare, i problemi riscontrati nell'effettuare \emph{matching} tra serie inter- e intra-dataset sono legati a:

\begin{itemize}
  \item
    inaccuratezza nel collocamento spaziale legata alla precisione numerica del dato registrato (generalmente accade per le stazioni meno recenti) e a conversioni del sistema di riferimento;
  \item
    prossimità di stazioni differenti, come avviene in particolare nei contesti cittadini;
  \item
    differenze nelle anagrafiche: dal semplice utilizzo di caratteri accentati o simboli all'impiego di nomi di località differenti;
  \item
    assenza dei codici identificativi univoci delle stazioni;
  \item
    differenze nella stima degli estremi giornalieri: estremi delle medie orarie o estremi assoluti giornalieri;
  \item
    presenza di serie già integrate con i dati di altre;
  \item
    differenti definizioni di ``giornata meteorologica'' come esposto in precedenza.
\end{itemize}

Queste osservazioni e la generale mancanza di una documentazione approfondita dei cataloghi che spieghi i criteri di raccolta dati, impongono l'elaborazione di metodi empirici e parametrici per determinare \emph{match} tra sequenze differenti. Si è scelto di utilizzare per la classificazione un approccio ad albero decisionale, che porta alla dichiarazione di \emph{match} avvenuto sottoponendo le coppie candidate ad una successione di test di confronto ad esito binario su parametri scelti in maniera ponderata. Partendo da considerazioni banali (due sequenze della stessa serie hanno dati uguali o ``simili'' ', sono ragionevolmente vicine e hanno anagrafiche simili) e confrontando dati e metadati per via grafica e tabulare si è giunti alla scelta del seguente set di parametri:

\begin{itemize}
  \item
    media delle differenze tra stime giornaliere, medie mensili e climatologie mensili prese con e senza valore assoluto;
  \item
    distanza sul piano tra le posizioni dichiarate;
  \item
    differenza tra le quote dichiarate;
  \item
    somiglianza tra nomi di stazione secondo l'algoritmo Jaro-Winkler;
  \item
    \(\mathrm{f}_0\): percentuale di stime giornaliere identiche al decimo di grado;
  \item
    percentuale di stime giornaliere non intere identiche al decimo di grado (alcune serie lunghe hanno registrato la temperatura in passato con precisione intera o semiintera: stazioni vicine hanno parte della sequenza identica per questa ragione);
  \item
    \(\mathrm{b}\): media dei segni delle differenze tra stime giornaliere diverse da 0 (permette di capire quanta parte delle misure di una stazione sono più estreme di quelle dell'altra);
  \item
    \(\mathrm{n_d}\): numero di giorni in cui entrambe le serie hanno misure valide.
\end{itemize}

I match sono stati cercati nell'insieme delle stazioni distanti al più \(15\:\mathrm{km}\) le une dalle altre. Sono stati calcolati i parametri introducendo nelle sequenze offset di -1, 0 e +1 giorni, per trovare match anche nelle situazioni in cui la stima è stata attribuita alla giornata precedente o successiva (cosa che capita ad esempio quando le definizioni di giornata meteorologica sono diverse).

Per ogni accoppiamento di dataset o di network di stazioni si sono fissati empiricamente dei valori-soglia per i parametri sopra elencati da utilizzare nei test di confronto dell'albero decisionale. Si è partiti dalle seguenti considerazioni:

\begin{itemize}
  \item
    valori rilevanti di \(\mathrm{f}_0\) (\(> \qty{15}{\percent}\)) a fronte di un numero significativo di giorni in comune (\(\mathrm{n_d} > 100\)) dovrebbero indicare un match ``orizzontale'';
  \item
    distanze ridotte (\(< 500\:\mathrm{m}\)) o valori di somiglianza tra anagrafiche alti (\(> \qty{90}{\percent}\)) potrebbero indicare sia match ``orizzontali'' che ``verticali'';
  \item
    valori rilevanti di \(\mathrm{f}_0\) associati a valori estremi di \(|\mathrm{b}|\) (\(\ge 0.8\)) e di segno discorde per le sequenze di minime e massime potrebbero indicare match tra serie con estremi assoluti a confronto con estremi delle medie.
\end{itemize}

Tali criteri sono stati adattati e integrati a seconda dei casi confrontando le analisi dei candidati match in tabelle. La bontà delle soglie scelte è stata valutata esaminando manualmente un campione rappresentativo dei risultati di ogni passaggio del merging controllando i grafici delle differenze tra valori delle stazioni accoppiate, le posizioni in anagrafica su Google Earth e tutti i casi in cui la media delle differenze fosse maggiore di \(\qty{0.5}{\degreeCelsius}\).

In alcune situazioni le soglie individuate non sono risultate sufficienti a stabilire i \emph{match} alla perfezione: si sono dovute fare integrazioni manuali sia per dichiarare situazioni di \emph{match} che per negare quelle trovate dalla procedura automatica.

\subsubsection{Unione delle sequenze}
L'ultimo passaggio del merging consiste nella combinazione di dati e metadati delle sequenze che compongono ogni serie. L'operazione viene effettuata integrando con gli elementi dei gruppi presi uno alla volta una serie master. Ciò richiede di stabilire criteri d'ordine e di metodo per l'unione sia dei metadati che dei dati.

Per quanto riguarda l'ordine di unione, tenendo conto delle osservazioni esposte nella @sec-sources, si è generalmente scelto di preferire i metadati delle agenzie regionali/provinciali a quelli dei dataset nazionali; per i dati invece si sono prese come riferimento le sequenze omogeneizzate ISAC-CNR, seguite in ordine da quelle delle agenzie regionali/provinciali, SCIA e infine DPC. Sia per i metadati che per i dati il secondo criterio d'ordine, usato nei casi di pareggio del primo, è la lontananza temporale dei dati forniti, con priorità data alle sequenze più recenti (che hanno generalmente metadati più accurati).

Per quanto riguarda invece il protocollo di integrazione delle stime giornaliere si è scelto di inserire iterativamente valori nella sequenza master, ove mancanti, se i contributi della sequenza integrante superano i due anni. L'aggiunta viene fatta sommando ai dati grezzi della serie integrante una correzione modellizzata in funzione del giorno dell'anno. Tale modello è scelto tra una serie di Fourier troncata al terzo ordine, uno scalare o lo zero a seconda della dimensione e della distribuzione del sample di anomalie:

\[
  \Delta T_{m,i}(d) \sim
  \begin{cases}
    \frac{a_0}{2} + \sum^3_{n=1} a_n \cos( 2\pi n t(d)) + b_n\sin(2\pi n t(d)) & \text{se } \mathcal{I}_{i,m} \ge 8 \\
    \frac{a_0}{2} & \text{se } 2 \le \mathcal{I}_{i,m} < 8 \\
    \qty{0}{\degreeCelsius} & \text{altrimenti} \\
  \end{cases}
\]

dove \(d\) è una data, \(t(d)\) il numero del giorno \(d\) normalizzato rispetto alla lunghezza del suo anno e \(\mathcal{I}_{i,m}\) il numero di mesi dell'anno diversi aventi almeno 20 anomalie valide.


\section{Controllo qualità}\label{ch:qc}
Il controllo qualità dei dati è stato sviluppato seguendo l'approccio descritto in~\cite{brunettiHighresolutionTemperatureClimatology2014}. Il procedimento è complesso a causa delle difficoltà derivanti dalla combinazione di misure provenienti sia da stazioni meccaniche che automatiche, come l'attribuzione della corretta giornata di misura e la gestione degli errori significativi legati ai malfunzionamenti episodici degli strumenti automatici.

Prima di procedere con il controllo qualità sono state eliminate le serie che forniscono pochi dati: meno di \(365 \cdot 5\) giorni di misure e meno di \(4 \cdot \mathrm{N_i}\) valori per ogni mese \(\mathrm{i}\) (dove \(\mathrm{N_i}\) è il numero di giorni del mese).

Il controllo qualità è stato suddiviso in tre fasi principali:
\begin{enumerate}
  \item identificazione degli errori grezzi e delle sequenze ripetute;
  \item confronto delle serie misurate con la rianalisi ERA5~\cite{hersbachERA5GlobalReanalysis2020};
  \item confronto con le serie limitrofe.
\end{enumerate}

Il passaggio al secondo punto è stato reso necessario dall'alto tasso di errore delle misure automatiche, che potrebbe compromettere il confronto diretto tra serie rilevate, oltre che per la ridotta efficacia del controllo con le serie limitrofe per le stazioni ai confini del dataset.

\paragraph{Identificazione degli errori grezzi}
In prima battuta, sono stati invalidati gli errori evidenti, rimuovendo tutte le misure con \(\lvert \mathrm{T} \rvert > \qty{50}{\degreeCelsius}\) e le sequenze di dati uguali per almeno sette giorni consecutivi (o intervallati da dati assenti).

\paragraph{Confronto con ERA5}
Le serie giornaliere del dataset sono state confrontate con serie sintetiche costruite utilizzando il metodo delle anomalie di Mitchell e Jones~\cite{mitchellImprovedMethodConstructing2005} e quello dei contributi relativi giornalieri di Di Luzio et al.~\cite{diluzioConstructingRetrospectiveGridded2008}.

Per ogni serie di minime e massime giornaliere, è stata costruita una sequenza mensile sintetica sommando alle climatologie 1961--1990 prodotte da Brunetti et al.~\cite{brunettiHighresolutionTemperatureClimatology2014} le anomalie delle medie mensili '91--'20 di ERA5 rispetto allo stesso periodo. Successivamente, sono stati calcolati i contributi relativi delle temperature giornaliere ERA5 al proprio mese come:
\[\mathrm{R_{m,i}} = \frac{\mathrm{T_{m,i}} - \mathrm{\overline{T}_m}}{\mathrm{\overline{T}_m}}\]
Questi contributi relativi sono stati interpolati sulle posizioni delle serie, e, utilizzando le nuove medie mensili sintetiche, si sono ottenuti i valori giornalieri sintetici finali, riconvertendo i contributi relativi in temperature assolute.

Il controllo qualità è stato quindi basato direttamente sul confronto tra le anomalie rispetto al proprio ciclo annuale delle serie sintetiche e delle serie misurate. Il ciclo non è stato calcolato direttamente come la media interannuale di ciascun giorno dell'anno disponibile, ma attraverso l'interpolazione delle prime tre armoniche, al fine di compensare le incertezze derivanti dalla ricostruzione delle normali. Per ogni coppia di anomalie relative alla stessa data (o alle date a distanza di un giorno in base a quale avesse la differenza minore, considerando i problemi di asincronia), si è dichiarata incompatibilità se la differenza in valore assoluto superava una soglia definita come \(\mathrm{Th_{ERA5}} = \mathrm{RMSE_{ERA5}}\cdot\mathrm{Th_0}\), troncata all'intervallo [\(\num{8}\)--\(\qty{16}{\degreeCelsius}\)]. Qui \(\mathrm{RMSE_{ERA5}}\) è la radice dell'errore quadratico medio delle anomalie della serie sintetica e \(\mathrm{Th_0}\) è una soglia scelta. L'intervallo di troncamento è stato scelto per evitare soglie troppo permissive in casi di serie altamente variabili o troppo selettive in casi opposti. Il controllo è stato ripetuto due volte, con soglie \(\mathrm{Th_0} = \qty{10}{\degreeCelsius}\) e \(\mathrm{Th_0} = \qty{5}{\degreeCelsius}\) al fine di rimuovere in un primo passo errori grossolani che avrebbero potuto compromettere la qualità del controllo stesso.

\paragraph{Confronto con serie limitrofe}\label{ch:close-series}
Per ogni serie giornaliera (``serie di test'') sono state selezionate sequenze limitrofe appropriate, con le quali è stata costruita una serie di riferimento. I valori della serie di test sono stati rimossi se risultavano più distanti di una certa soglia dai valori di riferimento.

Le serie limitrofe sono state selezionate tra quelle entro \qty{300}{\kilo\meter} di distanza orizzontale e \(\max(\qty{500}{\meter}, \mathrm{H_{test}}/2)\) di distanza verticale, con \(\mathrm{H_{test}}\) che rappresenta la quota della serie test.

Per ogni giorno con misura della serie di test è stato preso un set di valori da ciascuna delle serie limitrofe che soddisfaceva certi requisiti di completezza. La serie di riferimento doveva avere almeno \(\mathrm{N_{\min}}\) giorni validi in una finestra di \(\pm\mathrm{N_w}\) giorni centrata sul giorno di test, sia nell'anno corrispondente che negli \(\mathrm{N_y}\) anni precedenti e successivi. I parametri sono definiti come: \(\mathrm{N_{\min}} = \mathrm{N^0_{\min}} + INT(0.3\cdot\mathrm{N^0_{\min}}\cdot2\mathrm{N_y})\), con \(\mathrm{N^0_{\min}} = 15\), \(\mathrm{N_w} \in [25, 35]\) e \(\mathrm{N_y} \in \{1, 2\} \). Dal set di valori di riferimento giornalieri  è stato ricavato una set di valori ricostruiti riportando l'anomalia rispetto alla media dei riferimenti sulla finestra definita da \(\pm \mathrm{N_w}\) sulla media dei test sulla stessa finestra: \(\mathrm{T_{rec,i}} = \mathrm{T_{ref,i}} - \mathrm{\overline{T}_{ref}^W} + \mathrm{\overline{T}_{test}^W}\). Delle serie così ricostruite si sono prese al massimo le \(\mathrm{NR_{\max}} + 2\) (\(\mathrm{NR_{MAX}} = 15\)) selezionate in base alla correlazione più alta, considerando offset di -1, 0 e 1 giorno. Per ogni giorno, sono stati eliminati i valori più alti e più bassi. È stata infine presa come stima giornaliera ricostruita \(\mathrm{T_{rec,i}^{best}}\) la media ponderata dei valori rimanenti, usando come peso una funzione delle distanze orizzontale e verticale.

L'eliminazione dei valori di test è avvenuta nei casi in cui \(\lvert \mathrm{T_{rec,i}^{best}} - \mathrm{T_{test,i}} \rvert > \Delta_\mathrm{i}\), con \(\Delta_\mathrm{i} = \frac{1}{2}(\mathrm{T_{rec,i}^{\max} - T_{rec,i}^{\min}})\cdot\frac{1}{\overline{R^2}}\cdot5\frac{\mathrm{NR_{\max}}}{\mathrm{NR_{real}}}\) troncato all'intervallo [\(\num{4}\)--\(\qty{6}{\degreeCelsius}\)]. La soglia è dinamica e dipende dalla variabilità delle ricostruzioni, a livello giornaliero per mezzo della prima differenza e a livello globale tramite la media dei quadrati delle correlazioni \(\overline{R^2}\) e dalla disponibilità di serie per via di \(\frac{\mathrm{NR_{\max}}}{\mathrm{NR_{real}}}\), con \(\mathrm{NR_{real}}\) l'effettivo numero di serie tenute. A fronte dell'eliminazione di più di tre valori per mese è stato scartato tutto il mese.

\section{Omogeneizzazione}\label{ch:homo}
\input{chapters/stima_normali/omogeneizzazione}

\section{Aggregazione}\label{ch:aggregazione}
