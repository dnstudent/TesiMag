La seguente tesi illustra la costruzione di un dataset delle serie minime e massime giornaliere di temperatura atmosferica in prossimità dal suolo per il territorio del centro-nord Italia e delle regioni limitrofe. La priorità prefissata è raggiungere la massima copertura spaziale possibile di serie del trentennio 1991--2020 e un'alta qualità delle informazioni relative al collocamento geografico delle serie. Il lavoro aggiorna quanto presentato nello studio di Brunetti et al.\ nel 2014~\cite{brunettiHighresolutionTemperatureClimatology2014} per il trentennio 1961--1990.

Il risultato costituisce una buona base di dati per l'interpolazione di un dataset grigliato ad alta risoluzione delle normali climatiche mensili di temperatura (minima, massima e media) nel periodo e nell'area considerati.

Nella prima parte verrà illustrata la procedura di combinazione dei numerosi dataset forniti dai gestori delle reti di rilevazione meteorologica e da altri enti coinvolti. Nella seconda, invece, sarà presentata una valutazione preliminare della capacità di un modello di Local Weighted Linear Regression (LWLR) di interpolare i dati raccolti.