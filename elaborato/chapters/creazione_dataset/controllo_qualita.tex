Il controllo qualità dei dati è stato sviluppato seguendo l'approccio descritto in~\cite{brunettiHighresolutionTemperatureClimatology2014}. Il procedimento è complesso a causa delle difficoltà derivanti dalla combinazione di misure provenienti sia da stazioni meccaniche che automatiche, come l'attribuzione della corretta giornata di misura e la gestione degli errori significativi legati ai malfunzionamenti episodici degli strumenti automatici.

Prima di procedere con il controllo qualità sono state eliminate le serie che forniscono pochi dati: meno di \(365 \cdot 5\) giorni di misure e meno di \(4 \cdot \mathrm{N_i}\) valori per ogni mese \(\mathrm{i}\) (dove \(\mathrm{N_i}\) è il numero di giorni del mese).

Il controllo qualità è stato suddiviso in tre fasi principali:
\begin{enumerate}
  \item identificazione degli errori grezzi e delle sequenze ripetute;
  \item confronto delle serie misurate con la rianalisi ERA5~\cite{hersbachERA5GlobalReanalysis2020};
  \item confronto con le serie limitrofe.
\end{enumerate}

Il passaggio al secondo punto è stato reso necessario dall'alto tasso di errore delle misure automatiche, che potrebbe compromettere il confronto diretto tra serie rilevate, oltre che per la ridotta efficacia del controllo con le serie limitrofe per le stazioni ai confini del dataset.

\paragraph{Identificazione degli errori grezzi}
In prima battuta, sono stati invalidati gli errori evidenti, rimuovendo tutte le misure con \(\lvert \mathrm{T} \rvert > \qty{50}{\degreeCelsius}\) e le sequenze di dati uguali per almeno sette giorni consecutivi (o intervallati da dati assenti).

\paragraph{Confronto con ERA5}
Le serie giornaliere del dataset sono state confrontate con serie sintetiche costruite utilizzando il metodo delle anomalie di Mitchell e Jones~\cite{mitchellImprovedMethodConstructing2005} e quello dei contributi relativi giornalieri di Di Luzio et al.~\cite{diluzioConstructingRetrospectiveGridded2008}.

Per ogni serie di minime e massime giornaliere, è stata costruita una sequenza mensile sintetica sommando alle climatologie 1961--1990 prodotte da Brunetti et al.~\cite{brunettiHighresolutionTemperatureClimatology2014} le anomalie delle medie mensili '91--'20 di ERA5 rispetto allo stesso periodo. Successivamente, sono stati calcolati i contributi relativi delle temperature giornaliere ERA5 al proprio mese come:
\[\mathrm{R_{m,i}} = \frac{\mathrm{T_{m,i}} - \mathrm{\overline{T}_m}}{\mathrm{\overline{T}_m}}\]
Questi contributi relativi sono stati interpolati sulle posizioni delle serie, e, utilizzando le nuove medie mensili sintetiche, si sono ottenuti i valori giornalieri sintetici finali, riconvertendo i contributi relativi in temperature assolute.

Il controllo qualità è stato quindi basato direttamente sul confronto tra le anomalie rispetto al proprio ciclo annuale delle serie sintetiche e delle serie misurate. Il ciclo non è stato calcolato direttamente come la media interannuale di ciascun giorno dell'anno disponibile, ma attraverso l'interpolazione delle prime tre armoniche, al fine di compensare le incertezze derivanti dalla ricostruzione delle normali. Per ogni coppia di anomalie relative alla stessa data (o alle date a distanza di un giorno in base a quale avesse la differenza minore, considerando i problemi di asincronia), si è dichiarata incompatibilità se la differenza in valore assoluto superava una soglia definita come \(\mathrm{Th_{ERA5}} = \mathrm{RMSE_{ERA5}}\cdot\mathrm{Th_0}\), troncata all'intervallo [\(\num{8}\)--\(\qty{16}{\degreeCelsius}\)]. Qui \(\mathrm{RMSE_{ERA5}}\) è la radice dell'errore quadratico medio delle anomalie della serie sintetica e \(\mathrm{Th_0}\) è una soglia scelta. L'intervallo di troncamento è stato scelto per evitare soglie troppo permissive in casi di serie altamente variabili o troppo selettive in casi opposti. Il controllo è stato ripetuto due volte, con soglie \(\mathrm{Th_0} = \qty{10}{\degreeCelsius}\) e \(\mathrm{Th_0} = \qty{5}{\degreeCelsius}\) al fine di rimuovere in un primo passo errori grossolani che avrebbero potuto compromettere la qualità del controllo stesso.

\paragraph{Confronto con serie limitrofe}\label{ch:close-series}
Per ogni serie giornaliera (``serie di test'') sono state selezionate sequenze limitrofe appropriate, con le quali è stata costruita una serie di riferimento. I valori della serie di test sono stati rimossi se risultavano più distanti di una certa soglia dai valori di riferimento.

Le serie limitrofe sono state selezionate tra quelle entro \qty{300}{\kilo\meter} di distanza orizzontale e \(\max(\qty{500}{\meter}, \mathrm{H_{test}}/2)\) di distanza verticale, con \(\mathrm{H_{test}}\) che rappresenta la quota della serie test.

Per ogni giorno con misura della serie di test è stato preso un set di valori da ciascuna delle serie limitrofe che soddisfaceva certi requisiti di completezza. La serie di riferimento doveva avere almeno \(\mathrm{N_{\min}}\) giorni validi in una finestra di \(\pm\mathrm{N_w}\) giorni centrata sul giorno di test, sia nell'anno corrispondente che negli \(\mathrm{N_y}\) anni precedenti e successivi. I parametri sono definiti come: \(\mathrm{N_{\min}} = \mathrm{N^0_{\min}} + INT(0.3\cdot\mathrm{N^0_{\min}}\cdot2\mathrm{N_y})\), con \(\mathrm{N^0_{\min}} = 15\), \(\mathrm{N_w} \in [25, 35]\) e \(\mathrm{N_y} \in \{1, 2\} \). Dal set di valori di riferimento giornalieri  è stato ricavato una set di valori ricostruiti riportando l'anomalia rispetto alla media dei riferimenti sulla finestra definita da \(\pm \mathrm{N_w}\) sulla media dei test sulla stessa finestra: \(\mathrm{T_{rec,i}} = \mathrm{T_{ref,i}} - \mathrm{\overline{T}_{ref}^W} + \mathrm{\overline{T}_{test}^W}\). Delle serie così ricostruite si sono prese al massimo le \(\mathrm{NR_{\max}} + 2\) (\(\mathrm{NR_{MAX}} = 15\)) selezionate in base alla correlazione più alta, considerando offset di -1, 0 e 1 giorno. Per ogni giorno, sono stati eliminati i valori più alti e più bassi. È stata infine presa come stima giornaliera ricostruita \(\mathrm{T_{rec,i}^{best}}\) la media ponderata dei valori rimanenti, usando come peso una funzione delle distanze orizzontale e verticale.

L'eliminazione dei valori di test è avvenuta nei casi in cui \(\lvert \mathrm{T_{rec,i}^{best}} - \mathrm{T_{test,i}} \rvert > \Delta_\mathrm{i}\), con \(\Delta_\mathrm{i} = \frac{1}{2}(\mathrm{T_{rec,i}^{\max} - T_{rec,i}^{\min}})\cdot\frac{1}{\overline{R^2}}\cdot5\frac{\mathrm{NR_{\max}}}{\mathrm{NR_{real}}}\) troncato all'intervallo [\(\num{4}\)--\(\qty{6}{\degreeCelsius}\)]. La soglia è dinamica e dipende dalla variabilità delle ricostruzioni, a livello giornaliero per mezzo della prima differenza e a livello globale tramite la media dei quadrati delle correlazioni \(\overline{R^2}\) e dalla disponibilità di serie per via di \(\frac{\mathrm{NR_{\max}}}{\mathrm{NR_{real}}}\), con \(\mathrm{NR_{real}}\) l'effettivo numero di serie tenute. A fronte dell'eliminazione di più di tre valori per mese è stato scartato tutto il mese.