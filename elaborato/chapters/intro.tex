Lo scopo di questo lavoro è costruire un dataset degli estremi giornalieri di temperatura per il territorio del centro-nord Italia e delle regioni limitrofe, garantendo una buona copertura sia spaziale che temporale, con particolare attenzione al trentennio 1991--2020. Il lavoro aggiorna quanto presentato nello studio di Brunetti et al.\ nel 2014~\cite{brunettiHighresolutionTemperatureClimatology2014}.

Il risultato dovrebbe costituire una buona base di partenza per il calcolo di un dataset grigliato ad alta risoluzione delle normali climatiche mensili di temperatura (minima, massima e media) nel periodo e nell'area considerati.

Nella prima parte verrà illustrata la procedura di combinazione dei numerosi dataset forniti dai gestori delle reti di rilevazione meteorologica e da altri enti coinvolti. Nell seconda parte, invece, sarà presentata una valutazione preliminare della capacità del modello elaborato nello studio citato di interpolare i dati raccolti.

Si noti che esistono già dei prodotti che si prefiggono lo stesso scopo, primo tra tutti il sistema SCIA~\cite{climnormISPRA2022}, sviluppato dall'Istituto Superiore per la Protezione e la Ricerca Ambientale (ISPRA). Tuttavia, questo dataset presenta importanti limitazioni in termini di copertura spaziale (si confrontino le figure 1, 4 e 7 in~\cite[pp.~12--15]{climnormISPRA2022}) e temporale, tant'è che anche regioni dall'orografia complessa come la Lombardia e la Liguria risultano essere rappresentate da circa cinque stazioni l'una.
