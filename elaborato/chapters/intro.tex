I dataset a griglia di normali climatiche rivestono un ruolo cruciale nello studio dell’evoluzione del clima e come strumento di supporto in diversi ambiti produttivi e di gestione del territorio\cite{dalyGuidelinesAssessingSuitability2006}, con una prospettiva di lungo termine. La loro produzione si articola essenzialmente in due fasi: la costruzione delle serie di misure effettuate dagli strumenti distribuiti sul territorio e l'interpolazione su una griglia regolare alla risoluzione desiderata.

Lo sviluppo e la diffusione capillare delle stazioni meteorologiche automatizzate hanno reso disponibile una grande quantità di dati, caratterizzati da diversi livelli di rappresentatività locale, omogeneità temporale e tasso di errore. Se da un lato tali stazioni permettono la registrazione continua delle misure senza la supervisione di un operatore, migliorando potenzialmente la stima ad esempio degli estremi giornalieri di temperatura rispetto ai sistemi meccanici precedenti, dall'altro sono più soggette a malfunzionamenti, spesso rilevati solo a distanza di tempo. Inoltre, la varietà e l’abbondanza delle apparecchiature consentono una maggiore copertura del territorio, in particolare delle diverse quote geografiche, ma richiedono un impegno significativo per il monitoraggio e la gestione delle loro collocazioni fisiche.

In sostanza, come accade in svariati altri ambiti nell'epoca dei \emph{big data}, la quantità di informazioni disponibile si accompagna a una maggiore complessità nella loro gestione e validazione.

Dati georiferiti di qualità sono un requisito fondamentale per elaborare un modello in grado di interpolare valori rappresentativi. In particolare, su scala temporale climatologica e scala spaziale regionale, è noto~\cite{dalyKnowledgebasedApproachStatistical2002} da tempo che la temperatura dell'aria in prossimità del suolo, variabile meteorologica studiata in questo lavoro, è fortemente correlata all'elevazione, alla distanza da grandi corpi d'acqua (mare e laghi di dimensioni significative) e dall'orientazione geografica della superficie del terreno. Di conseguenza, quanto più fine è la risoluzione richiesta per un dataset su griglia, tanto più accurate devono essere tali informazioni quando usate per addestrare un modello di interpolazione.

Esistono già, per il territorio italiano, dei prodotti che si prefiggono lo scopo di fornire un dataset delle massime e minime giornaliere nel periodo considerato, primo tra tutti il sistema SCIA~\cite{desiatoSCIASystemBetter2007a}, sviluppato dall'Istituto Superiore per la Protezione e la Ricerca Ambientale (ISPRA). Tuttavia, questo dataset presenta importanti limitazioni in termini di copertura spaziale (si confrontino le figure 1, 4 e 7 in~\cite[pp.~12--15]{climnormISPRA2022}) e temporale, tant'è che anche regioni dall'orografia complessa come la Lombardia e la Liguria risultano essere rappresentate da un numero di stazioni inferiore alla decina.
