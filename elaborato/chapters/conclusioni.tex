Il lavoro ha portato alla costruzione di un dataset con un'ampia e capillare copertura spaziale e temporale di dati, significativamente superiore rispetto a quanto realizzato in altri studi. Permane tuttavia un fondamentale problema di accuratezza e completezza di metadati quali lo storico di ricollocamento delle stazioni e il posizionamento delle stazioni meno recenti, che i metodi di controllo e correzione utilizzati non possono determinare in maniera precisa. Lo storico in particolare darebbe informazioni molto preziose ai fini di una più efficace procedura di omogeneizzazione dei dati, che per ora si è limitata ad eliminare le misure e sostituirle con stime tramite \emph{gap filling}. Inoltre sarebbe utile approfondire la differenza tra le statistiche di errore italiane e quelle dei territori limitrofi, che potrebbe essere legata alla qualità dei dati sottostanti.

L'analisi preliminare delle capacità del modello sviluppato nel lavoro precedente di interpolare le climatologie, ha evidenziato che ci sono una serie di aspetti e relazioni geospaziali che ancora non sono ben comprese dalla procedura, in particolare la relazione locale con la quota e la distanza dal mare.

I risultati, tuttavia, considerando la poca attenzione dedicata al \emph{fine tuning} del modello, sono promettenti: il lavoro verrà portato avanti studiando come migliorare il modello per far fronte alle problematiche evidenziate. Inoltre, considerata l'accuratezza raggiunta per il posizionamento delle stazioni moderne, verrà sperimentato l'utilizzo di un DEM a risoluzione più fine per provare a caratterizzare meglio i parametri geografici impiegati dal modello.
Nel frattempo il lavoro è stato esteso con le stesse procedure al resto del territorio italiano, con risultati simili.