Il lavoro ha portato alla costruzione di un dataset con un'ampia e capillare copertura spaziale e temporale di dati, significativamente superiore rispetto a quanto realizzato in altri studi. Permangono tuttavia alcune criticità, tra cui la mancanza di uno studio di omogeneità, che risulta complesso per un dataset di tali dimensioni e con una disponibilità limitata dei metadati utili a una valutazione adeguata.

L'analisi preliminare delle capacità del modello sviluppato nel lavoro precedente di interpolare le climatologie, ha evidenziato che ci sono una serie di aspetti e relazioni geospaziali che ancora non sono ben comprese dalla procedura, in particolare la relazione locale con la quota e la distanza dal mare.