Le normali climatiche degli estremi giornalieri di temperatura si calcolano come media sui trent'anni 1991--2020, per ciascuno dei dodici mesi, delle medie mensili delle temperature minime e massime giornaliere.

La stima delle normali a partire dai dati controllati richiede un protocollo di gestione dei dati mancanti. Le linee guida dell'Organizzazione Meteorologica Mondiale (WMO)~\cite{WMOGuidelinesNormalsCalculation2017} raccomandano di non calcolare stime mensili quando mancano più di dieci giorni in totale o più di quattro giorni consecutivi, poiché tali lacune possono influenzare in modo significativo la statistica. Tuttavia, è considerata una buona pratica completare le lacune, ad esempio tramite interpolazione. In questo studio, si è scelto di effettuare il completamento dei dati mancanti (\emph{gap filling}) interpolando, ove possibile, i dati delle serie limitrofe (serie di riferimento) a scala giornaliera. Qualora ciò non fosse possibile, si è optato per l'interpolazione dei valori delle serie mensili.

Il \emph{gap filling} giornaliero è stato effettuato utilizzando le stime \(\mathrm{T^{best}_{rec,i}}\) calcolate nel paragrafo precedente.
Il completamento a scala mensile, invece, è stato realizzato tramite due metodi: il metodo del completamento ``contemporaneo'' e quello del completamento ``storico'', entrambi basati sulle serie limitrofe.

Il metodo contemporaneo si applica a serie (serie di test) che, nelle mensilità problematiche, dispongono di almeno cinque giorni utili e almeno quattro serie vicine con misure negli stessi giorni. In questo caso, si stima una correzione da apportare alla media mensile della serie di test, calcolata sui giorni disponibili, per riportarla al valore reale del mese. Tale correzione si ottiene ricavando lo scarto tra il valore medio reale del mese e il valore medio dei giorni in comune con la serie di test, utilizzando le \(\mathrm{N} < 9\) serie di riferimento selezionate. Si assume che tale differenza presenti una variabilità spaziale ridotta, permettendo così di prendere come correzione la media del campione individuato.

Il metodo storico, invece, si applica quando la serie da completare dispone, negli anni del trentennio, di almeno cinque stime mensili in comune con serie limitrofe che contengano il mese mancante. In questo caso, per ciascuna serie di riferimento si calcola lo scarto tra il valore del mese mancante e la media dei valori dei mesi in comune. Tale scarto viene quindi sommato alla media dei mesi in comune della serie da completare. Anche in questo caso, si assume che la differenza tra la serie da ricostruire e la serie di riferimento sia costante nel tempo. Come miglior stima, si utilizza la mediana delle nove serie con il peso geografico più elevato rispetto alla serie da completare.