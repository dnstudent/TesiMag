% Options for packages loaded elsewhere
\PassOptionsToPackage{unicode}{hyperref}
\PassOptionsToPackage{hyphens}{url}
\documentclass[
]{article}
\usepackage{xcolor}
\usepackage[margin=1in]{geometry}
\usepackage{amsmath,amssymb}
\setcounter{secnumdepth}{-\maxdimen} % remove section numbering
\usepackage{iftex}
\ifPDFTeX
  \usepackage[T1]{fontenc}
  \usepackage[utf8]{inputenc}
  \usepackage{textcomp} % provide euro and other symbols
\else % if luatex or xetex
  \usepackage{unicode-math} % this also loads fontspec
  \defaultfontfeatures{Scale=MatchLowercase}
  \defaultfontfeatures[\rmfamily]{Ligatures=TeX,Scale=1}
\fi
\usepackage{lmodern}
\ifPDFTeX\else
  % xetex/luatex font selection
\fi
% Use upquote if available, for straight quotes in verbatim environments
\IfFileExists{upquote.sty}{\usepackage{upquote}}{}
\IfFileExists{microtype.sty}{% use microtype if available
  \usepackage[]{microtype}
  \UseMicrotypeSet[protrusion]{basicmath} % disable protrusion for tt fonts
}{}
\makeatletter
\@ifundefined{KOMAClassName}{% if non-KOMA class
  \IfFileExists{parskip.sty}{%
    \usepackage{parskip}
  }{% else
    \setlength{\parindent}{0pt}
    \setlength{\parskip}{6pt plus 2pt minus 1pt}}
}{% if KOMA class
  \KOMAoptions{parskip=half}}
\makeatother
\usepackage{graphicx}
\makeatletter
\newsavebox\pandoc@box
\newcommand*\pandocbounded[1]{% scales image to fit in text height/width
  \sbox\pandoc@box{#1}%
  \Gscale@div\@tempa{\textheight}{\dimexpr\ht\pandoc@box+\dp\pandoc@box\relax}%
  \Gscale@div\@tempb{\linewidth}{\wd\pandoc@box}%
  \ifdim\@tempb\p@<\@tempa\p@\let\@tempa\@tempb\fi% select the smaller of both
  \ifdim\@tempa\p@<\p@\scalebox{\@tempa}{\usebox\pandoc@box}%
  \else\usebox{\pandoc@box}%
  \fi%
}
% Set default figure placement to htbp
\def\fps@figure{htbp}
\makeatother
\setlength{\emergencystretch}{3em} % prevent overfull lines
\providecommand{\tightlist}{%
  \setlength{\itemsep}{0pt}\setlength{\parskip}{0pt}}
\usepackage{bookmark}
\IfFileExists{xurl.sty}{\usepackage{xurl}}{} % add URL line breaks if available
\urlstyle{same}
\hypersetup{
  pdftitle={Elaborazione di un dataset degli estremi di temperatura mensili per il territorio italiano},
  pdfauthor={Davide Nicoli},
  hidelinks,
  pdfcreator={LaTeX via pandoc}}

\title{Elaborazione di un dataset degli estremi di temperatura mensili
per il territorio italiano}
\author{Davide Nicoli}
\date{}

\begin{document}
\maketitle

\subsection{Introduzione}\label{introduzione}

Il lavoro ha avuto l'obiettivo di elaborare un dataset delle normali
climatologiche mensili relative al trentennio 1991-2020 delle
temperatura minime e massime giornaliere per le serie registrate nel
centro-nord italia, nell'ambito di un più ampio progetto volto alla
costruzione di un dataset ad alta risoluzione esteso all'intero
territorio italiano. In particolare si cerca di aggiornare

Il dataset risultante è stato costruito attraverso la raccolta, il
controllo e l'integrazione di dati provenienti da diverse fonti
provinciali, regionali e nazionali.

\subsection{Metodologia}\label{metodologia}

\begin{enumerate}
\def\labelenumi{\arabic{enumi}.}
\tightlist
\item
  Raccolta dei dati: sono stati acquisiti dati da vari enti, tra cui
  ARPA, SCIA (dataset ISPRA), CNR-ISAC e il dataset del Dipartimento di
  Protezione Civile, prestando particolare attenzione a coprire il
  periodo 1991-2020. La frammentazione e l'eterogeneità delle fonti
  hanno richiesto un attento lavoro di standardizzazione;
\item
  Merging: le serie di dati sono state integrate attraverso un processo
  che ha combinato fonti diverse per garantire una maggiore copertura
  spaziale e temporale ed eliminare le serie duplicate;
\item
  Controllo qualità: è stata eseguita un'analisi dei dati per
  individuare errori e incongruenze, utilizzando confronti con la
  rianalisi ERA5 e con stazioni meteorologiche limitrofe tramite
  tecniche appositamente studiate in altri lavori;
\item
  Stima delle normali climatiche: sono state calcolate medie
  climatologiche mensili, considerando solo le serie con almeno cinque
  anni di dati validi e completando quelle con un numero non troppo
  elevato di dati mancanti.
\end{enumerate}

\subsection{Risultati}\label{risultati}

\begin{itemize}
\tightlist
\item
  Il dataset finale comprende 2508 serie, con un miglioramento
  significativo rispetto alle fonti originali in termini di continuità e
  accuratezza.
\item
  È stata riscontrata una discrepanza maggiore nelle temperature minime
  rispetto alle massime, particolarmente evidente in aree montane e
  costiere.
\item
  L'analisi ha evidenziato la necessità di ulteriori miglioramenti nel
  trattamento dei dati di regioni con orografia complessa.
\end{itemize}

\subsection{Conclusioni}\label{conclusioni}

L'integrazione delle diverse fonti ha permesso di ottenere un dataset
più completo e affidabile per lo studio del clima italiano. Tuttavia,
permangono alcune criticità, come la difficoltà di omogeneizzare dati
provenienti da strumenti e metodologie differenti. Il lavoro rappresenta
un passo avanti nella disponibilità di dati climatologici per analisi
ambientali e previsioni a lungo termine.

\end{document}
